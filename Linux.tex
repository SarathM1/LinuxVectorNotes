%%%%%%%%%%%%%%%%%%%%%%%%%%%%%%%%%%%%%%%%%
% The Legrand Orange Book
% LaTeX Template
% Version 2.1.1 (14/2/16)
%
% This template has been downloaded from:
% http://www.LaTeXTemplates.com
%
% Original author:
% Mathias Legrand (legrand.mathias@gmail.com) with modifications by:
% Vel (vel@latextemplates.com)
%
% License:
% CC BY-NC-SA 3.0 (http://creativecommons.org/licenses/by-nc-sa/3.0/)
%
% Compiling this template:
% This template uses biber for its bibliography and makeindex for its index.
% When you first open the template, compile it from the command line with the 
% commands below to make sure your LaTeX distribution is configured correctly:
%
% 1) pdflatex main
% 2) makeindex main.idx -s StyleInd.ist
% 3) biber main
% 4) pdflatex main x 2
%
% After this, when you wish to update the bibliography/index use the appropriate
% command above and make sure to compile with pdflatex several times 
% afterwards to propagate your changes to the document.
%
% This template also uses a number of packages which may need to be
% updated to the newest versions for the template to compile. It is strongly
% recommended you update your LaTeX distribution if you have any
% compilation errors.
%
% Important note:
% Chapter heading images should have a 2:1 width:height ratio,
% e.g. 920px width and 460px height.
%
%%%%%%%%%%%%%%%%%%%%%%%%%%%%%%%%%%%%%%%%%

%----------------------------------------------------------------------------------------
%	PACKAGES AND OTHER DOCUMENT CONFIGURATIONS
%----------------------------------------------------------------------------------------

\documentclass[11pt,fleqn]{book} % Default font size and left-justified equations

%----------------------------------------------------------------------------------------

\input{structure} % Insert the commands.tex file which contains the majority of the structure behind the template

\begin{document}

%----------------------------------------------------------------------------------------
%	TITLE PAGE
%----------------------------------------------------------------------------------------
\usechapterimagefalse
\begingroup
\thispagestyle{empty}
\begin{tikzpicture}[remember picture,overlay]
\coordinate [below=12cm] (midpoint) at (current page.north);
\node at (current page.north west)
{\begin{tikzpicture}[remember picture,overlay]
\node[anchor=north west,inner sep=0pt] at (0,0) {}; % Background image
\draw[anchor=north] (midpoint) node [fill=ocre!30!white,fill opacity=0.6,text opacity=1,inner sep=1cm]{\Huge\centering\bfseries\sffamily\parbox[c][][t]{\paperwidth}{\centering Linux System Programming\\[15pt] % Book title
{\Large Using C}\\[20pt] % Subtitle
{\huge Sarath M}}}; % Author name
\end{tikzpicture}};
\end{tikzpicture}
\vfill
\endgroup


%----------------------------------------------------------------------------------------
%	TABLE OF LISTINGS
%----------------------------------------------------------------------------------------
\lstlistoflistings 

%----------------------------------------------------------------------------------------
%	TABLE OF CONTENTS
%----------------------------------------------------------------------------------------

\usechapterimagefalse % If you don't want to include a chapter image, use this to toggle images off - it can be enabled later with \usechapterimagetrue

%\chapterimage{chapter_head_1.pdf} % Table of contents heading image

\pagestyle{empty} % No headers

\tableofcontents % Print the table of contents itself

\cleardoublepage % Forces the first chapter to start on an odd page so it's on the right

\pagestyle{fancy} % Print headers again

%----------------------------------------------------------------------------------------
%	PART
%----------------------------------------------------------------------------------------

% % \part{Part One}

%----------------------------------------------------------------------------------------
%	CHAPTER 1
%----------------------------------------------------------------------------------------

\chapterimage{chapter_head_2.pdf} % Chapter heading image

\chapter{Introduction}

\section{Topics}

\begin{enumerate}
	\item Process Management
	\item File and file management
	\item Memory and memory management
	\item Signals and Signal handling
	\item Thread and Thread management
	\item Inter Process Communication
	\item Process Synchronisation
	\item Shell Scripitng
\end{enumerate}

\section{Defenition's}

\begin{description}
\item[Operating System] OS is a resource manager/allocater (rather than a mere interface between the user and hardware) which is responsible for managing the resources which is connected to the CPU

\item[BIOS] Basic Input Output Systems
When we switch ON the system, BIOS program is executed. First job of BIOS is to check if basic input output are connected or not. After that the BIOS executes a program called Bootloader

\item[Bootloader] picks ups an OS from hard disk and loads it into RAM. Load time/Boot time is the time taken for this operation.

\item[Grub loader] Boot loader in Linux
\item[NT loader] Boot loader in Windows
\end{description}

\section{OS in Embedded Systems}
\textit{Why OS in embedded systems?}

Without OS only one program can be run at a time

With OS multiple process can be run simultaneously. Thus performance of the product increases.

\section{Components of an OS}
\begin{enumerate}
	\item Application
	\item Services
\end{enumerate}

\begin{description}
 \item[Appication] are optional services. Applications runs only when we intentionally run it. 
 \item[Services] are mandatory. Kernel starts executing when OS is loaded into RAM 
 \item[Kernel] All the services combined is called a kernel
\end{description}

\begin{myfigure}{os}{OS}{0.8}
\end{myfigure}
%----------------------------------------------------------------------------------------
%	CHAPTER 2
%----------------------------------------------------------------------------------------

\chapter{Process Management}\index{Process Management}

\begin{description}
	\item [Process] Thre program which is in execution is called as the process.To execute a file say a.out, a copy of a.out is loaded into RAM
	\item [Process Manager] - Manages the different process. 
\end{description}


\begin{myremark}{}
\textit{\textbf{ps -e}}\index{ps command} Command to display the processes which are currently running
\end{myremark}

\section{Process Manager}
For every process, the process manager will provide a process ID, ie; the process manager identifies each process with a process ID.

\begin{myremark}{}
\begin{itemize}
	\item ./a.out \& : If we want to run a command in background
	
	\item fg : command to bring the background process to foreground
	
	\item fg <jobId>: To move a specific process to foregraound
	
	\item ps -e | grep pts/0: If we want to list all programs running in terminal \textit{pts/0}

\end{itemize}
\end{myremark}

\begin{figure}[h]
\centering\includegraphics[scale=0.5]{background}
\caption{Output}
\end{figure}

\begin{description}
	\item[Shell] - Command Interpreter. When user wants to interact with the OS we use the shell. \textbf{ex: bash}
\end{description}


\begin{myremark}{}
kill: command used to send signals to a particular process

ex: kill -9 1769, where 1769 is process id

\end{myremark}


%------------------------------------------------

\section{PID and PPID}\index{getpid()}\index{getppid()}

\begin{mycode}{programs/getpid.c}
\end{mycode}

\begin{myremark}{}
\begin{itemize}
	\item \textbf{get\_pid()}: returns the process id
	\item \textbf{get\_ppid()}: return the parent process id 
\end{itemize}
\end{myremark}

The parent is nothing but the bash. Whenever a new terminal is opened, a bash process is creatted. This bash is the parent of the program being executed
%------------------------------------------------

\section{System() function}\index{system()}

\begin{mycode}{programs/system.c}
\end{mycode}

\begin{myoutput}{programs/system.txt}
\end{myoutput}

system() function executes a command specified by 'command' by calling \textbf{\texttt{/bin/sh -c command}} and returns after the command has been completed.

\subsection*{Example}

Consider the following program

\begin{mycode}{programs/example.c}
\end{mycode}

Suppose we created a new executable for this code name \textbf{p1.out}

Now we can run the above program in p2.c using system() as shown below:

\begin{mycode}{programs/p2.c}
\end{mycode}

\begin{myoutput}{programs/example1.txt}
\end{myoutput}

when \textit{ctr+C} is pressed Hai will since only then control returns from \textit{p1} because of \textit{while(1)}

\subsection*{Example}

Consider the following program, How many processes are created by exampl2.c

\begin{mycode}{programs/example2.c}
\end{mycode}

\textbf{Answer: 6} 

\begin{enumerate}
	\item Bash
	\item a.out
	\item sh
	\item cal
	\item sh
	\item ls
\end{enumerate}

\begin{myfigure}{system}{Output}{0.8}
\end{myfigure}

\section{Context Switch}\index{Context Switch}

Whenever a process is loaded into the RAM the process manager creates one table called \textbf{Process Control Block} which consists of information about the process.

\begin{myfigure}{PCB}{Process Control Block}{1}
\end{myfigure}

Process manager stores the process information in linked list fashion.

When the process is executing the process information is loaded into CPU. For every process a particular time (time slice) is given. When the time slice expires the process manager loads the updated process information to the respective PCB and the next PCB is loaded into the CPU.
\linebreak\linebreak
This loading and unloading of information is called \textbf{\textit{Context Switch}}

When rhe process execution is complete the PCB is destoyed
%------------------------------------------------

\section{fork() function}\index{fork()}

\begin{myremark}{
pid\_t fork(void)}
\end{myremark}

\begin{itemize}
	\item Used to create a child process
	\item Creates a new process by duplicating the calling process
	\item The new process referred to as child process which is exact duplicate of calling process referred to as parent
\end{itemize}

\subsection*{Example}

\begin{mycode}{programs/fork1.c}
\end{mycode}

\begin{myoutput}{programs/fork1.txt}
\end{myoutput}

\begin{myremark}{
All the statements after the fork() are executed twice!!
\linebreak Once by the \textit{parent} and then by the \textit{child}
\linebreak Parent executes in the foreground, whereas child executes in the background !!}
\end{myremark}

\subsection{Race Condition}
Normally the parent process gets the CPU time after creating the child


But when multiple processes are waiting for CPU time a race condition exists and the order of execution of these processes may vary

\subsection*{Example}

\begin{mycode}{programs/fork2.c}
\end{mycode}

\begin{myoutput}{programs/fork2.txt}
\end{myoutput}

\begin{myfigure}{fork2_demo}{Explanation}{0.1}
\end{myfigure}

Here we can't predict the order of the processes

\section*{Note}

Upon successful creation of a child fork returns child process id in a parent and zero in a child. On failure fork() returns -1 and no child is created \& errno is set appropriately.

\begin{mycode}{programs/fork3.c}
\end{mycode}

\begin{myoutput}{programs/fork3.txt}
\end{myoutput}

\subsection{Seperating child and parent code using if-else}

\begin{mycode}{programs/fork4.c}
\end{mycode}

\begin{myoutput}{programs/fork4.txt}
\end{myoutput}

Here child executes the if part and parent exuctes the else part. All code outside if, else is executed by both child and parent

\subsubsection*{Excercise}

Write a program to create 3 child process from the same parent

\begin{mycode}{programs/fork5.c}
\end{mycode}

\begin{myoutput}{programs/fork5.txt}
\end{myoutput}

\begin{myfigure}{fork3}{Structure}{0.5}
\end{myfigure}

%----------------------------------------------------------------------------------------
%	BIBLIOGRAPHY
%----------------------------------------------------------------------------------------

\chapter*{Bibliography}
\addcontentsline{toc}{chapter}{\textcolor{ocre}{Bibliography}}
\section*{Books}
\addcontentsline{toc}{section}{Books}
\printbibliography[heading=bibempty,type=book]
\section*{Articles}
\addcontentsline{toc}{section}{Articles}
\printbibliography[heading=bibempty,type=article]

%----------------------------------------------------------------------------------------
%	INDEX
%----------------------------------------------------------------------------------------

\cleardoublepage
\phantomsection
\setlength{\columnsep}{0.75cm}
\addcontentsline{toc}{chapter}{\textcolor{ocre}{Index}}
\printindex

%----------------------------------------------------------------------------------------

\end{document}
